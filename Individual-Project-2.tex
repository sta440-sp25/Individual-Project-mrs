% Options for packages loaded elsewhere
\PassOptionsToPackage{unicode}{hyperref}
\PassOptionsToPackage{hyphens}{url}
%
\documentclass[
  4pt,
]{article}
\usepackage{amsmath,amssymb}
\usepackage{iftex}
\ifPDFTeX
  \usepackage[T1]{fontenc}
  \usepackage[utf8]{inputenc}
  \usepackage{textcomp} % provide euro and other symbols
\else % if luatex or xetex
  \usepackage{unicode-math} % this also loads fontspec
  \defaultfontfeatures{Scale=MatchLowercase}
  \defaultfontfeatures[\rmfamily]{Ligatures=TeX,Scale=1}
\fi
\usepackage{lmodern}
\ifPDFTeX\else
  % xetex/luatex font selection
\fi
% Use upquote if available, for straight quotes in verbatim environments
\IfFileExists{upquote.sty}{\usepackage{upquote}}{}
\IfFileExists{microtype.sty}{% use microtype if available
  \usepackage[]{microtype}
  \UseMicrotypeSet[protrusion]{basicmath} % disable protrusion for tt fonts
}{}
\makeatletter
\@ifundefined{KOMAClassName}{% if non-KOMA class
  \IfFileExists{parskip.sty}{%
    \usepackage{parskip}
  }{% else
    \setlength{\parindent}{0pt}
    \setlength{\parskip}{6pt plus 2pt minus 1pt}}
}{% if KOMA class
  \KOMAoptions{parskip=half}}
\makeatother
\usepackage{xcolor}
\usepackage[margin=0.1in]{geometry}
\usepackage{longtable,booktabs,array}
\usepackage{calc} % for calculating minipage widths
% Correct order of tables after \paragraph or \subparagraph
\usepackage{etoolbox}
\makeatletter
\patchcmd\longtable{\par}{\if@noskipsec\mbox{}\fi\par}{}{}
\makeatother
% Allow footnotes in longtable head/foot
\IfFileExists{footnotehyper.sty}{\usepackage{footnotehyper}}{\usepackage{footnote}}
\makesavenoteenv{longtable}
\usepackage{graphicx}
\makeatletter
\def\maxwidth{\ifdim\Gin@nat@width>\linewidth\linewidth\else\Gin@nat@width\fi}
\def\maxheight{\ifdim\Gin@nat@height>\textheight\textheight\else\Gin@nat@height\fi}
\makeatother
% Scale images if necessary, so that they will not overflow the page
% margins by default, and it is still possible to overwrite the defaults
% using explicit options in \includegraphics[width, height, ...]{}
\setkeys{Gin}{width=\maxwidth,height=\maxheight,keepaspectratio}
% Set default figure placement to htbp
\makeatletter
\def\fps@figure{htbp}
\makeatother
\setlength{\emergencystretch}{3em} % prevent overfull lines
\providecommand{\tightlist}{%
  \setlength{\itemsep}{0pt}\setlength{\parskip}{0pt}}
\setcounter{secnumdepth}{-\maxdimen} % remove section numbering
\makeatletter
\renewcommand{\maketitle}{
  \begin{center}
    {\large \@title \par}
  \end{center}
}
\makeatother
\ifLuaTeX
  \usepackage{selnolig}  % disable illegal ligatures
\fi
\usepackage{bookmark}
\IfFileExists{xurl.sty}{\usepackage{xurl}}{} % add URL line breaks if available
\urlstyle{same}
\hypersetup{
  pdftitle={Michael Shoichet - Individual Project},
  hidelinks,
  pdfcreator={LaTeX via pandoc}}

\title{Michael Shoichet - Individual Project}
\author{}
\date{\vspace{-2.5em}}

\begin{document}
\maketitle

\subparagraph{Introduction}\label{introduction}

The analysis of this project may provide insight into relationships
across quality of life for Durham residents. As well, it may help city
administrators make decisions on improving certain areas within Durham
and the city's strategic growth plan. The dataset comes from the 2022
City of Durham Resident Survey, an annual survey administered by the ETC
Institute. A random sample of households in the city of Durham were
mailed a survey, cover letter, and postage-paid return envelope.
Questions to the survey included opinions on many facets of Durham's
amenities, including quality of schooling, police protection, library
access, condition of streets, etc. The original dataset featured 891
unique and independent observations from 226 questions from the survey.

The key objectives of this project are to understand a potential
association between quality of life and number of years that a resident
has lived in Durham. Furthermore, this project attempts to recognize a
relationship within quality of life based on the Durham residents'
homeowner status, which may have specific location grouping factors
within Durham (Zip Code). From examining previous research from a study
of 79 European cities, it was determined that living in a city for
increasing thresholds of 5 years result in higher levels of resident
satisfaction (Weziak-Bialowolska 2016). Analysis of this dataset will
see if similar conclusions can be drawn.

\subparagraph{Relevant Variables}\label{relevant-variables}

The response variable of interest was \texttt{quality}, a response to
part 6 of question 3 (Q3-6) from the 2022 Resident Survey. This variable
was the response to the residents' overall quality of life in Durham,
with response options of (1) ``very dissatisfied'', (2)
``dissatisfied'', (3) ``neutral'', (4) ``satisfied'', and (5) ``very
satisfied''. The other variables included in the project analysis were
\texttt{ID} (identification number of the resident), \texttt{years}, a
response to question 27 (Q27) of the survey indicating how many years
the resident has lived in Durham, \texttt{income}, a response to
question 33 (Q33) of the survey indicating total annual household income
with response options of (1) under \$30,000, (2) \$30,000 - \$59,999,
(3) \$60,000 - \$99,999, or (4) \$100,000 or more, \texttt{house}, a
response to question 30 (Q30) of the survey indication if you (1) own or
(2) rent your house, and \texttt{Zip}, the Zip Code of the residents'
address within Durham.

For data cleaning, all rows of relevant variables that were unreported
or reported as a value of 9 (N/A) were removed from analysis. Certain
variables used in analysis (\texttt{quality}, \texttt{years},
\texttt{income} and \texttt{house}) were renamed as such from their
original question number, which allowed for more comprehensive model
interpretation. The variables \texttt{quality}, \texttt{income}, and
\texttt{house} were mutated to factors to account for their leveled
responses, \texttt{Zip} was mutated to a factor to be used as a random
intercept grouping variable in the mixed model, and \texttt{years} was
mutated to be a continuous numerical variable. This left 759
observations.

\subparagraph{Methods}\label{methods}

To assess the first objective of understanding a possible association
between quality of life and length of time a resident has lived in
Durham, an ordinal logistic regression model was included. This is
because while the \texttt{quality} response variable has natural order
of responses numbered 1-5, the intervals between levels are not
guaranteed to be equal. For example, the difference between ``very
dissatisfied'' and ``dissatisfied'' (1-2) may not be the same as the
difference between ``neutral'' and ``satisfied'' (3-4). The proportional
odds assumption was met, with the associated summary table output shown
in Table 1. Seeing as quality of life is not solely dependent on the
number of years living in a city, a second ordinal model was included to
understand the interaction effect that years lived in Durham has with
household income. The proportional odds and multicollinearity assumption
was met, with the associated summary table output shown in Table 2. To
assess the second objective of potential relationship based on home
ownership status and location of residence, a cumulative link model was
included. This allowed for understanding the ordinal response variable
\texttt{quality} based on the interaction effect of years lived in
Durham and home ownership status with the random effect of Zip Code. The
multicollinearity assumption was met, with associated summary table
outputs shown in Tables 3 and 4.

\subparagraph{Results}\label{results}

The results from the first ordinal logistic regression model suggests
that, holding else constant, each additional year a resident has lived
in Durham is associated with 0.993 (e\^{}-0.007) times the odds of being
in a higher satisfactory category. This means that the odds of reporting
greater satisfaction decrease by 0.7\% for an additional year a resident
lives in Durham. With a p-value of 0.037 which is less than the
\(\alpha = 0.05\) significance level, this provides statistically
significant evidence that the number of years a resident lives in Durham
is associated with their reported quality of life level (Table 1).
Adding household income level as an interaction effect proved to not be
statistically significant as all the \texttt{years\ x\ income} p-values
were greater than the \(\alpha = 0.05\) level. Holding else constant,
residents in income level 3 (\$60,000 - \$99,999) are predicted to be
1.976 (e\^{}0.681) times more likely to report a higher satisfactory
category compared to the baseline income level 1 (under \$30,000).
Holding else constant, residents in income level 4 (\$100,000 or more)
are predicted to be 2.184 (e\^{}0.781) times more likely to report a
higher satisfactory category compared to the baseline income level 1
(under \$30,000). With p-values of 0.050 and 0.019 respectively, which
is less than the \(\alpha = 0.05\) level, this provides statistically
significant evidence that income level is associated with a residents'
reported quality of life level (Table 2).

The results from the cumulative link mixed model show that none of the
fixed effects or the interaction effect between number of years lived in
Durham and home ownership status were statistically meaningful in its
association with quality of life after accounting for the Zip Code.
However, the standard error of 0.300 for Zip Code suggests the
underlying satisfaction intercept varies slightly by neighborhood (Table
3 and 4).

\subparagraph{Discussion}\label{discussion}

Based on the results, the models were not very effective at explaining
the relationships between quality of life in Durham and the data from
the residents. The large p-values caused few interpretable predictor
variables, and the interpretation for \texttt{years} in Table 1
conflicted with previous research. A potential issue with the data
collection is that only those interested in completing the survey were
the ones recorded, it was not a mandatory response. This could lead to
biased results. Asking these types of questions within the census, or
another form of required response would potentially increase reliability
of the models and interpretability. Future work could analyze much more
of the factors that were within the survey among a residents' lifestyle
in understanding quality of life (access to schooling, public safety,
affordability, maintenance, etc).

\subparagraph{Appendix}\label{appendix}

\begin{longtable}[]{@{}lllll@{}}
\caption{Ordinal Logistic Regression Summary}\tabularnewline
\toprule\noalign{}
& Value & Standard Error & t-value & p-value \\
\midrule\noalign{}
\endfirsthead
\toprule\noalign{}
& Value & Standard Error & t-value & p-value \\
\midrule\noalign{}
\endhead
\bottomrule\noalign{}
\endlastfoot
years & -0.007 & 0.004 & -2.084 & 0.037 \\
1\textbar2 & -3.775 & 0.245 & -15.393 & \textless0.001 \\
2\textbar3 & -1.861 & 0.136 & -13.687 & \textless0.001 \\
3\textbar4 & -0.752 & 0.118 & -6.397 & \textless0.001 \\
4\textbar5 & 1.477 & 0.13 & 11.34 & \textless0.001 \\
\end{longtable}

\begin{longtable}[]{@{}lllll@{}}
\caption{Ordinal Logistic Regression (Years × Income
Interaction)}\tabularnewline
\toprule\noalign{}
& Value & Standard Error & t-value & p-value \\
\midrule\noalign{}
\endfirsthead
\toprule\noalign{}
& Value & Standard Error & t-value & p-value \\
\midrule\noalign{}
\endhead
\bottomrule\noalign{}
\endlastfoot
years & -0.005 & 0.007 & -0.675 & 0.500 \\
income2 & 0.336 & 0.354 & 0.947 & 0.344 \\
income3 & 0.681 & 0.347 & 1.962 & 0.050 \\
income4 & 0.781 & 0.333 & 2.341 & 0.019 \\
years:income2 & 0 & 0.01 & -0.037 & 0.970 \\
years:income3 & 0 & 0.011 & -0.024 & 0.981 \\
years:income4 & 0.005 & 0.011 & 0.446 & 0.656 \\
1\textbar2 & -3.235 & 0.342 & -9.471 & \textless0.001 \\
2\textbar3 & -1.305 & 0.274 & -4.766 & \textless0.001 \\
3\textbar4 & -0.171 & 0.267 & -0.64 & 0.522 \\
4\textbar5 & 2.106 & 0.281 & 7.485 & \textless0.001 \\
\end{longtable}

\begin{longtable}[]{@{}lrrrl@{}}
\caption{Fixed Effects: Mixed Ordinal Logistic
Regression}\tabularnewline
\toprule\noalign{}
Term & Estimate & Standard Error & t-value & p-value \\
\midrule\noalign{}
\endfirsthead
\toprule\noalign{}
Term & Estimate & Standard Error & t-value & p-value \\
\midrule\noalign{}
\endhead
\bottomrule\noalign{}
\endlastfoot
1\textbar2 & -3.795 & 0.283 & -13.407 & \textless0.001 \\
2\textbar3 & -1.864 & 0.196 & -9.521 & \textless0.001 \\
3\textbar4 & -0.737 & 0.183 & -4.018 & \textless0.001 \\
4\textbar5 & 1.531 & 0.193 & 7.945 & \textless0.001 \\
years & -0.006 & 0.005 & -1.218 & 0.223 \\
house2 & -0.004 & 0.231 & -0.017 & 0.986 \\
years:house2 & -0.004 & 0.008 & -0.589 & 0.556 \\
\end{longtable}

\begin{longtable}[]{@{}lllrrr@{}}
\caption{Random Effects Summary (Zip Code)}\tabularnewline
\toprule\noalign{}
& Group & Effect & Variance & Standard.Error & Groups \\
\midrule\noalign{}
\endfirsthead
\toprule\noalign{}
& Group & Effect & Variance & Standard.Error & Groups \\
\midrule\noalign{}
\endhead
\bottomrule\noalign{}
\endlastfoot
(Intercept) & Zip & Intercept & 0.09016 & 0.3003 & 14 \\
\end{longtable}

\includegraphics{Individual-Project-2_files/figure-latex/tables-figures-1.pdf}

Despite not having a largely significant impact on reported quality of
life, the heatmap shows the proportion of homeowner staus based on the
specific Zip Codes within Durham.

\subparagraph{Sources:}\label{sources}

\begin{itemize}
\tightlist
\item
  \url{https://live-durhamnc.opendata.arcgis.com/documents/2459a4fcb09345b1a6135321bf91eac9/about}
  - data source
\item
  \url{https://www.sciencedirect.com/science/article/pii/S0264275116301330}
  - Weziak-Bialowolska research
\item
  \url{https://r-graph-gallery.com/heatmap} - heat map guidance
\item
  \url{https://search.r-project.org/CRAN/refmans/tigris/html/zctas.html}
  - download zip code tabulation area (ZTCA)
\end{itemize}

\end{document}
